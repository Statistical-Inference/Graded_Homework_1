\documentclass[11pt]{article}
\usepackage{amsmath}
\usepackage{amssymb}
\usepackage{algpseudocode,algorithm}
\usepackage{graphicx}
\usepackage{psfrag,color}
\usepackage{fullpage}
\usepackage{epsfig}
\usepackage{graphicx}

\setlength{\topmargin}{-0.7in}
\setlength{\textwidth}{6.5in}
\setlength{\oddsidemargin}{0.0in}
\setlength{\textheight}{10.0in}
\setlength{\parindent}{0in}

\renewcommand{\baselinestretch}{1.2}
\renewcommand\arraystretch{1.5}
\newcommand{\problem}[1]{ \medskip \pp $\underline{\rm Problem\ #1}$\\ }


\pagestyle{empty}

\def\pp{\par\noindent}


\begin{document}

\begin{flushright}
{\bf STAT GR5703---Spring 2020}
\end{flushright}
\begin{flushleft}
Group: Zining Fan, Mutian Wang, Siyuan Wang\\
UNI: zf2234, mw3386, sw3418\\
\end{flushleft}

\bigskip
\centerline{\bf Graded Homework 1- Exercise 1}

\bigskip 
\begin{enumerate}
\item
As the probability density function of d is:
\[f(d;\lambda) = \lambda e^{-\lambda d}\]
To get a $p^{th}$ population quantitle:
\[\int_{0}^{x} \lambda e^{-\lambda x} dx = p\]
\[-\frac{1}{\lambda} e^{-\lambda x} + \frac{1}{\lambda} e^0= \frac{p}{\lambda}\]
So the $p^{th}$ population quantitle is:
\[x = Q_D(p) = -\frac{\ln(1-p)}{\lambda}\]
\item
From the expectation of exponential distribution:
\[E(D) = \frac{1}{\lambda}\]
\[Q_D(p) = -\ln(1-p) E(D)\]
The first empirical moment is: 
\[\hat{\mu}_1 = \bar{D_n}\]
\[\hat{Q_D(p)}^{MM}_1 = \varphi^{-1}(\hat{\mu}_1)\]
\[= -\ln(1-p) \bar{D_n}\]

\item
From CLT we know that:
\[\bar{D_n} \sim N(\frac{1}{\lambda},\frac{1}{n\lambda^2})\]
\[-\ln(1-p) \bar{D_n} \sim N(-\frac{\ln(1-p)}{\lambda},\frac{(\ln(1-p))^2}{n\lambda^2})\]
\[\frac{-\ln(1-p) \bar{D_n} - -\frac{\ln(1-p)}{\lambda}}{\frac{-\ln(1-p)}{\sqrt{n}\lambda}}\sim N(0,1)\]
The margin should be:
\[Z_{1-\frac{\alpha}{2}} \frac{-\ln(1-p)}{\sqrt{n} \lambda}\]
So the confidence interval should be:
\[[-\ln(1-p) \bar{D_n} -  Z_{1-\frac{\alpha}{2}} \frac{-\ln(1-p)}{\sqrt{n} \lambda},-\ln(1-p) \bar{D_n} +  Z_{1-\frac{\alpha}{2}} \frac{-\ln(1-p)}{\sqrt{n} \lambda}]\]


\item
As $D_i \sim exp(\lambda)$,
$\lambda D_i \sim exp(1)$, which can be written as:
\[\lambda D_i \sim gamma(1,1)\]
\[\sum \limits_{i=1}^{n} \lambda D_i \sim gamma(n,1)\]
\[\frac{1}{n}\sum \limits_{i=1}^{n} \lambda D_i \sim gamma(n,n)\]
so that:
\[\lambda \bar{D_n} \sim gamma(n,n)\]
As we know that $\lambda = \frac{\ln{2}}{Q_p(0.5)}$, we could construct the interval:
\[gamma_{\frac{\alpha}{2}}(n,n) \leq \frac{\ln2 \bar{D_n}}{Q_p(0.5)} \leq gamma_{1-\frac{\alpha}{2}(n,n)}\]
Get that:
\[\frac{\ln{2}\bar{D_n}}{gamma_{1-\frac{\alpha}{2}}(n,n)} \leq \hat{Q_p(0.5)} \leq \frac{\ln{2}\bar{D_n}}{gamma_{\frac{\alpha}{2}}(n,n)}\]
So the exact confidence interval should be:
\[[\frac{\ln{2}\bar{D_n}}{gamma_{1-\frac{\alpha}{2}}(n,n)},\frac{\ln{2}\bar{D_n}}{gamma_{\frac{\alpha}{2}}(n,n)}]\]

\bigskip 
\end{enumerate}
\end{document}
